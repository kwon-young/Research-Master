%%% HEAD
\documentclass[a4paper]{article}

\usepackage{fullpage} % Package to use full page
\usepackage{parskip} % Package to tweak paragraph skipping
%\usepackage{tikz} % Package for drawing
\usepackage{amsmath}
\usepackage{hyperref}
\usepackage[utf8]{inputenc}
\usepackage[T1]{fontenc}
\usepackage{lmodern}
\usepackage[french]{babel}

\usepackage{graphicx}

\title{Ontologie de la notation musicale}
\author{Arthur Le Guennec \and Kwon-Young Choi}
\date{\normalsize\today}
%%% END HEAD

\begin{document}
\maketitle

\section{Ontologies : definition}

Une ontologie est un ensembole structuré de terme et de relations permettant d'expliciter formellement les connaissances spécifiques lié à un domaine.

* grosse expension ces dernières années
* existe des ontologies sur tout et n'importe quoi
* une ontologie permet de partager des connaissances sur un domaine entre des personnes et des agents virtuels
* une ontologie est réutilisable
* une ontologie permet de séparer la connaissance lié à un domaine et la connaissance lié au logiciel
* permet d'analyser les connaissances d'un domaine.

Il existe plusieurs language pour définir une ontologie.
Dans le cardre du Web sémantique, le W3C (World Wide Web Consortium) a énormément travaillé à développer et standardiser l'utilisation des ontologies.
Nous allons d'ailleurs utiliser ici le language OWL (Web Ontology Language) qui s'appuie sur le standard RDF et la syntaxe XML et qui a été spécifié par le W3C.
* une ontologie est constitué de classes et de sous-classes
* les relations sont explicités avec des triplet RDF

\section{Spécification d'une ontologie sur la notation musicale}

Une ontologie est utilisé pour expliciter les connaissances lié à un domaine spécifique.
Nous avons choisi ici de réaliser une ontologie pour expliciter la notation musicale.
La musique occidentale utilise une grammaire bidimensionnelle qui permet de décrire la hauteur, le rythme et le tempo d'une musique.
Cela permet la sauvegarde et le partage de la musique.

La notation musicale est donc constitué d'un ensemble de symboles organisés selon une grammaire bidimensionnelle.
On qualifie cette grammaire de bidimensionnelle car le temps est symbolisé par la progression de la partition de droite à gauche et la synchronisation temporelle est exprimé par l'alignement vertical des symboles.
De plus, l'organisation d'une partition musicale est hiérarchique.
En effet, une partition est constitué de portée, qui contient elle-même des mesures.
La différente hauteur des notes est modélisé par cinq lignes.
Chaque lignes et interlignes correspondent à une hauteur de notes particulières.

La notation musicale utilisent une grande variété de symboles, même si ces symboles restent assez simples dans leur formes.
On trouve dans la figure ? un résumé de la plupart des symboles musicaux.
Durant ce travail, nous avons voulu représenté grâce à une ontologie l'organisation d'une partition musicale très simple.
Nous avons pris comme exemple de modéliser la partition de "J'ai du bon tabac dans ma tabatière".
Nous avons donc tout d'abord déclaré des classes de symboles comme les clefs qui contient des sous-classes qui représentent les différentes clefs utilisées dans la notation musicale comme la clef de sol.
La notion de temps est très important dans une partition musical.


\end{document}
